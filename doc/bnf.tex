
\documentclass[a4paper]{article}

\usepackage[utf8]{inputenc}
\usepackage[T1]{fontenc}
\usepackage{geometry}
%\usepackage[francais]{babel}

\usepackage{color}
\usepackage{listings}
\lstset{%
language=Python,                % choose the language of the code
basicstyle=\footnotesize,       % the size of the fonts that are used for the code
numbers=left,                   % where to put the line-numbers
numberstyle=\footnotesize,      % the size of the fonts that are used for the line-numbers
stepnumber=1,                   % the step between two line-numbers. If it is 1 each line will be numbered
numbersep=5pt,                  % how far the line-numbers are from the code
backgroundcolor=\color{white},  % choose the background color. You must add \usepackage{color}
showspaces=false,               % show spaces adding particular underscores
showstringspaces=false,         % underline spaces within strings
showtabs=false,                 % show tabs within strings adding particular underscores
frame=single,                   % adds a frame around the code
tabsize=2,              % sets default tabsize to 2 spaces
captionpos=b,                   % sets the caption-position to bottom
breaklines=true,        % sets automatic line breaking
breakatwhitespace=false,    % sets if automatic breaks should only happen at whitespace
escapeinside={\%}{)}          % if you want to add a comment within your code
}


\pagestyle{headings}


\title{Parsing language with \texttt{bnf} package}
\author{Londeix Raphaël}


\begin{document}

    \maketitle

    \begin{abstract}
        The \texttt{bnf} package is a small python package that
        makes possible to easily parse any language with a BNF
        notation. This document explains how to do it, what you
        can and can't do, and details advanced uses.
    \end{abstract}

    \newpage
    \section{The \texttt{bnf} package}
        \subsection{A concept of token}
            When you write any parser, you basicaly need to define
            \textit{tokens}. A token is a well defined part of your
            input language, definable with BNF rules. It can be a
            terminal one or not. We will see simple tokens classes
            before combining them. We also will detail different
            ways to define your tokens. But in the first hand,
            we need to introduce basic context class.

            \subsubsection{Source of data}
                In order to introduce tokens and parsing methods, we need
                a source of data. I called that object a context, because
                it aggregates a bunch of things, commonly grouped as a context.
                \texttt{bnf} package provide a simple, ready to use, context
                class. The context class is used by all tokens as a source
                while matching patterns. The \texttt{bnf.Context} class
                is created with a filename, and tokens will implicitly read
                that file.

                Initializing a context object is really simple as:
                \lstinputlisting{listings/bnf/ex00.py}

                You are encouraged to define your own context class, but for
                now, we just need the default one. A more complete working
                example is given in the next section.

            \subsubsection{\texttt{Literal} class}
                The simplest token match a string, a \textit{literal}. We
                will define later all possible ways to use tokens, so just take
                following examples as they are, examples.
                So, lets parse our first hello world:
                \lstinputlisting{listings/bnf/ex01.py}

                I won't write again the startup part of this script, and focus on BNF
                declaration. If you need more explanation on Python \_\_main\_\_ scope,
                consider learn it before any other step. So, I put a comment
                just before the most interesting line. It's really helpfull to have
                such comments before every defined token. The next line declare
                \texttt{language} as a literal. The only file that can be parsed a this point
                is a file that contains exactly the string "Hello, World!" and nothing else.
                In facts, it will accept whitespaces before or after that string, but we will
                detail whitespaces rules in a few moment.
                Before you ask, every token class has a \texttt{parse} method, that takes
                a context.

            \subsubsection{A more useful notation}
                The previous example does not allow any customization of matching process.
                There are other ways to obtain the same result, but with more flexibility.
                \lstinputlisting{listings/bnf/ex02.py}

                This is exactly the same example. Subclassing default token class is really
                simple. You just need to know special member or method names and override
                default behaviour. The \texttt{bnf.Literal} class has a special member
                \texttt{\_\_token\_\_}, which contains the string to match.
                We need to have an instance of the \texttt{Language} class, just because
                of the need to call the parse method. It's generally not needed to instanciate
                token classes as simple as this one.
                Let's do something when the string matches:
                \lstinputlisting{listings/bnf/ex03.py}

                Ok, still not useful, but you can see the point. When any token match,
                the method \texttt{onMatch} is called.

        \subsection{Group and alternative}
            We cannot do lots of things without a concept of group. A \textit{group} in BNF
            makes us able to say that a token can be present more than on time. We suppose
            you have basic skill in BNF notation.

            \subsubsection{Groups by example}
                Let's take a bunch of simple BNF example and write the equivalent in Python:
                \lstinputlisting{listings/bnf/ex04.py}



        \subsection{Making life easier}

    \newpage
    \section{Parse any language}
        \subsection{The language container}
        \subsection{Separating the concepts from the needs}
        \subsection{}

    \newpage
    \section{Advanced uses}
        \subsection{Your own token classes}
        \subsection{Overriding the match method}
        \subsection{}

\end{document}

