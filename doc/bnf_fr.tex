
\documentclass[a4paper]{article}

\usepackage[utf8]{inputenc}
\usepackage[T1]{fontenc}
\usepackage{geometry}
\usepackage[francais]{babel}

\usepackage{color}
\usepackage{listings}

\lstset{%
language=Python,                % choose the language of the code
basicstyle=\footnotesize,       % the size of the fonts that are used for the code
numbers=left,                   % where to put the line-numbers
numberstyle=\footnotesize,      % the size of the fonts that are used for the line-numbers
stepnumber=1,                   % the step between two line-numbers. If it is 1 each line will be numbered
numbersep=5pt,                  % how far the line-numbers are from the code
backgroundcolor=\color{white},  % choose the background color. You must add \usepackage{color}
showspaces=false,               % show spaces adding particular underscores
showstringspaces=false,         % underline spaces within strings
showtabs=false,                 % show tabs within strings adding particular underscores
frame=single,                   % adds a frame around the code
tabsize=2,              % sets default tabsize to 2 spaces
captionpos=b,                   % sets the caption-position to bottom
breaklines=true,        % sets automatic line breaking
breakatwhitespace=false,    % sets if automatic breaks should only happen at whitespace
escapeinside={\%}{)}          % if you want to add a comment within your code
}


\pagestyle{headings}


\title{Le package \texttt{bnf}}
\author{Londeix Raphaël}

\newcommand{\insertpython}[1]{%
{\ttfamily\lstinputlisting{#1}}%
}

\renewcommand\rmdefault{pbk}  % famille à utiliser pour du Roman
%\renewcommand\sfdefault{cmss} % famille à utiliser pour du Sans Serif
%\renewcommand\ttdefault{bmtt} % famille à utiliser pour du «machine à écrire»
%\renewcommand\bfdefault{bx}   % collection à utiliser pour du gras
%\renewcommand\mddefault{m}    % collection à utiliser pour du moyen
%\renewcommand\itdefault{it}   % forme à utiliser pour de l'italique
%\renewcommand\sldefault{sl}   % forme à utiliser pour du penché
%\renewcommand\scdefault{sc}   % forme à utiliser pour de petites majuscules
%\renewcommand\updefault{n}    % forme à utiliser pour du droit
%\renewcommand\encodingdefault{OT1}      % codage normal
\renewcommand\familydefault{\rmdefault} % famille normale: Roman
%\renewcommand\seriesdefault{\mddefault} % collection normale: moyen
%\renewcommand\shapedefault{\updefault}  % forme normale: droit

\newcommand{\fixed}[1]{\texttt{#1}}
\newcommand{\bnf}{B.N.F.}

\begin{document}

    \maketitle

    \begin{abstract}
        Le package \fixed{bnf} fournit plusieurs classes permettant
        l'écriture de \textit{parser} la plus proche possible de la
        BNF sous-jacente.
    \end{abstract}

    \section{Introduction}
        Un besoin récurrent en informatique est l'extraction de contenu,
        qui n'est qu'en fait de la traduction. la \bnf (Backus Naur Form)
        permet de décrire la grammaire d'un language avec une syntaxe simple.
        Celle-ci permet de relier entre eux des \textit{tokens},
        parties du language donné, afin de former une définition globale
        de celui-ci. La syntaxe et le caractère dynamique de Python permettent
        une grande flexibilité lors de l'implémentation. J'utiliserai des
        abusivement certains anglissisme comme parser pour désigner
        la hiérarchie des classes qui compose le parser, matching lorsqu'il
        s'agira du processus d'identification des tokens.


    \section{La source de donnée}
        Pour pouvoir comment définir les tokens, nous avons besoin d'une
        source de données. J'ai appelé cet objet le \textit{contexte} car
        il aggrège un certains nombre d'objet liés à l'état courant. le
        package \fixed{bnf} fournit une classe simple et prête à l'emploi.
        Celle-ci est utilisée par tous les tokens comme une source lors
        du \textit{matching}. Une instance de la classe \fixed{bnf.Context} est
        construites avec un nom de fichier en argument, fichier qui sera
        lu par les tokens.

        Initialiser un contexte est aussi simple que ça:
        \insertpython{listings/bnf/ex01.py}

        Vous noterez que je me force à mettre la notation \bnf avant chaque
        définition, cela se révèle très utile lors de la relecture. La ligne
        suivante déclare language comme étant un \textit{littéral}, que la
        prochaine section décrit. Le seul fichier pouvant être parsé doit
        contenir exactement la chaîne "Hello, World!". En fait, il va
        accepter des caractères blancs avant et/ou après, mais nous verrons
        après comment influencer l'extraction de ces caractères.

    \section{Le concept de token}
    \section{Les chaînes de caractères}
    \section{Les groupes et les alternatives}
    \section{Les classes utiles}

\end{document}
