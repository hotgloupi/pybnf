
\documentclass[a4paper]{article}

\usepackage[utf8]{inputenc}
\usepackage[T1]{fontenc}
\usepackage{geometry}
\usepackage[francais]{babel}

\usepackage{color}
\usepackage{listings}

\lstset{%
language=Python,                % choose the language of the code
basicstyle=\footnotesize,       % the size of the fonts that are used for the code
numbers=left,                   % where to put the line-numbers
numberstyle=\footnotesize,      % the size of the fonts that are used for the line-numbers
stepnumber=1,                   % the step between two line-numbers. If it is 1 each line will be numbered
numbersep=5pt,                  % how far the line-numbers are from the code
backgroundcolor=\color{white},  % choose the background color. You must add \usepackage{color}
showspaces=false,               % show spaces adding particular underscores
showstringspaces=false,         % underline spaces within strings
showtabs=false,                 % show tabs within strings adding particular underscores
frame=single,                   % adds a frame around the code
tabsize=2,              % sets default tabsize to 2 spaces
captionpos=b,                   % sets the caption-position to bottom
breaklines=true,        % sets automatic line breaking
breakatwhitespace=false,    % sets if automatic breaks should only happen at whitespace
escapeinside={\%}{)}          % if you want to add a comment within your code
}


\pagestyle{headings}


\title{Le package \texttt{bnf}}
\author{Londeix Raphaël}

\newcommand{\insertpython}[1]{%
{\ttfamily\lstinputlisting{#1}}%
}

\renewcommand\rmdefault{pbk}  % famille à utiliser pour du Roman
%\renewcommand\sfdefault{cmss} % famille à utiliser pour du Sans Serif
%\renewcommand\ttdefault{bmtt} % famille à utiliser pour du «machine à écrire»
%\renewcommand\bfdefault{bx}   % collection à utiliser pour du gras
%\renewcommand\mddefault{m}    % collection à utiliser pour du moyen
%\renewcommand\itdefault{it}   % forme à utiliser pour de l'italique
%\renewcommand\sldefault{sl}   % forme à utiliser pour du penché
%\renewcommand\scdefault{sc}   % forme à utiliser pour de petites majuscules
%\renewcommand\updefault{n}    % forme à utiliser pour du droit
%\renewcommand\encodingdefault{OT1}      % codage normal
\renewcommand\familydefault{\rmdefault} % famille normale: Roman
%\renewcommand\seriesdefault{\mddefault} % collection normale: moyen
%\renewcommand\shapedefault{\updefault}  % forme normale: droit

\newcommand{\fixed}[1]{\texttt{#1}}
\newcommand{\bnf}{B.N.F.}

\begin{document}

    \maketitle

    \begin{abstract}
        Le package \fixed{bnf} fournit plusieurs classes permettant
        l'écriture de \textit{parser} la plus proche possible de la
        BNF sous-jacente.
    \end{abstract}

    \newpage
    \section{Introduction}
        Un besoin récurrent en informatique est l'extraction de contenu,
        qui n'est qu'en fait de la traduction. la \bnf (Backus Naur Form)
        permet de décrire la grammaire d'un language avec une syntaxe simple.
        Celle-ci permet de relier entre eux des \textit{tokens},
        parties du language donné, afin de former une définition globale
        de celui-ci. La syntaxe et le caractère dynamique de Python permettent
        une grande flexibilité lors de l'implémentation. Cette librairie se
        base sur cette flexibilité pour décrire la grammaire d'un language.
        Nous verrons en détails les classes de bases fournies et leurs
        utilisations. J'utiliserai
        abusivement certains anglissisme comme \emph{parser} pour désigner
        la hiérarchie des classes qui compose le parser, ou bien \emph{matching}
        lorsqu'il s'agira du processus d'identification des tokens.

    \newpage
    \section{Fonctionnement général}
        \subsection{La source de donnée}
            Pour pouvoir définir et utiliser les tokens, nous avons besoin d'une
            source de données. J'ai appelé cet objet le \emph{contexte} car
            il aggrège un certains nombre d'objet liés à l'état courant. le
            package \fixed{bnf} fournit une classe simple et prête à l'emploi.
            Celle-ci est utilisée par tous les tokens comme une source lors
            du \textit{matching}. Une instance de la classe \fixed{bnf.Context} est
            construites avec un nom de fichier en argument, fichier qui sera
            lu par les tokens.

            Initialiser un contexte est aussi simple que ça~:
            \insertpython{listings/bnf/ex00.py}

            Un exemple fonctionnel pourrait donner ceci~:
            \insertpython{listings/bnf/ex01.py}

            Vous noterez que je me force à mettre la notation \bnf avant chaque
            définition, cela se révèle très utile lors de la relecture. La ligne 10
            déclare \fixed{language} comme étant un \emph{littéral}, que la
            prochaine section décrit. Le seul fichier pouvant être parsé doit
            contenir exactement la chaîne "Hello, World!". En fait, il va
            accepter des caractères blancs avant et/ou après. Les tokens sont
            configuré pour extraires les caractères blanc usuels par défaut.
            Nous verrons comment modifier ce comportement, notemment lorsqu'il
            s'agira d'ignorer autre chose que des caractères (des commentaires
            par exemple).

        \subsection{Le concept de token}
            La classe \fixed{bnf.Token} défini les attributs et méthodes que
            doivent impérativement posséder ou redéfinir tous les tokens.
            La lecture du code source peut-être utile puisqu'il ne contient
            que très peu de code, et beaucoup de commentaire.
            Comme dit précédemment, la définition
            d'une grammaire résulte en une structure arborescente, la racine
            étant le language en lui-même, les feuilles étant les règles
            terminales de la \bnf.

            Chaque token possède au moins une méthode \fixed{match}, prenant
            le contexte en argument et renvoyant un booléen. La valeur de ce booléen
            informe l'appelant que le token à \emph{matché} ou pas. L'écriture de
            cette méthode impose plusieurs contraintes afin que l'ensemble puisse
            fonctionner harmonieusement.
            Voici le comportement attendu par la méthode match~:
            \begin{itemize}
                \item Ne retourne rien d'autre que \fixed{True} ou \fixed{False}
                \item Laisse le contexte inchangé lorsque le retour est \fixed{False}
                \item Consomme effectivement les caractères lorsque le retour est \fixed{True}
                \item Appelle la méthode \fixed{onBeginMatch} avant toute autre chose
                \item Appelle la méthode \fixed{onMatch} lorsque le retour est \fixed{True}
                \item Appelle la méthode \fixed{onEndMatch} juste avant de retourner.
            \end{itemize}

            \noindent
            Les méthodes \fixed{onBeginMatch}, \fixed{onMatch} et
            \fixed{onEndMatch} ne font rien par défaut, elles doivent pouvoir être
            surdéfinies sans modifier le comportement général du token. Chacune
            de ces méthodes prend le contexte en argument, et \fixed{onEndMatch}
            prend en plus le résultat du matching en plus.

            Pour des raisons de commodité, tous les tokens possèdes une méthode
            \fixed{\_\_str\_\_} qui permet d'afficher les tokens dans leur
            représentation la plus fidèle possible à la \bnf.

    \newpage
    \section{Les classes de base}
        Nous allons ici décrire les classes les plus utiles pour créer un parseur.
        Tout d'abord, nous allons décrire les classes permettant de créer les
        règles terminales, puis les classes représentant les groupes (au sens de
        la \bnf) et enfin les classes qui nous simplifierons la vie.

        \subsection{Les chaînes de caractères}
            Nous avons abordé le concept de \emph{littéral} dans la première
            section. La classe utilisée ici est \fixed{bnf.Literal}, elle permet
            de \emph{matcher} une chaîne de caractère fixe. Reprennons le précédent
            exemple en utilisant une syntaxe un tout petit peu différente~:
            \insertpython{listings/bnf/ex02.py}
            Les lignes de 10 à 14 seront omisent dans les examples suivants,
            nous nous concentrerons sur la \bnf. Ici, nous surclassons la
            classe \fixed{bnf.Literal}, ce qui ici n'apporte rien, mais sera
            la voie privilégiée par la suite. En effet, c'est elle
            qui permet de redéfinir les \emph{hooks} de matching
            (\fixed{onBeginMatch}, \fixed{onMatch} et \fixed{onEndMatch}).
            Les classes possèdent certains attributs spéciaux à connaître.
            Heureusement, ceux-ci sont en nombre très réduit et permettent
            d'accelerer grandement la surdefinition des classes. Pour la classe
            \fixed{bnf.Literal}, c'est l'attribut \fixed{\_\_token\_\_} qui
            contient la chaîne à matcher.

        \subsection{Les groupes et les alternatives}
        \subsection{Les classes utiles}

    \newpage
    \section{Le parser en condition réelle}
        \subsection{Les groupes récursifs}
        \subsection{Surclasser le contexte}
        \subsection{Vos propres tokens}

\end{document}
