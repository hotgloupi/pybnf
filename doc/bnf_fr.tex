
\documentclass[a4paper]{article}

\usepackage[utf8]{inputenc}
\usepackage[T1]{fontenc}
\usepackage{geometry}
\usepackage[francais]{babel}

% \usepackage{listings}
% \usepackage{color}
% \usepackage{xcolor}
% \usepackage{framed}
%
% \lstloadlanguages{Python}
% \lstset{%
% basicstyle=\footnotesize\ttfamily,       % the size of the fonts that are used for the code
% numbers=left,                   % where to put the line-numbers
% numberstyle=\footnotesize,      % the size of the fonts that are used for the line-numbers
% stepnumber=1,                   % the step between two line-numbers. If it is 1 each line will be numbered
% numbersep=5pt,                  % how far the line-numbers are from the code
% backgroundcolor=\color{white},  % choose the background color. You must add \usepackage{color}
% showspaces=false,               % show spaces adding particular underscores
% showstringspaces=false,         % underline spaces within strings
% showtabs=false,                 % show tabs within strings adding particular underscores
% frame=single,                   % adds a frame around the code
% tabsize=2,              % sets default tabsize to 2 spaces
% captionpos=b,                   % sets the caption-position to bottom
% breaklines=true,        % sets automatic line breaking
% breakatwhitespace=false,    % sets if automatic breaks should only happen at whitespace
% stringstyle=\itshape
% %escapeinside={\%}{)}          % if you want to add a comment within your code
% }

% Python listing setup

\usepackage{color}
\usepackage[procnames]{listings}
\usepackage{textcomp}
\usepackage{setspace}
\usepackage{palatino}
\renewcommand{\lstlistlistingname}{Code Listings}
\renewcommand{\lstlistingname}{Code Listing}
\definecolor{gray}{gray}{0.5}
\definecolor{green}{rgb}{0,0.5,0}
\definecolor{lightgreen}{rgb}{0,0.7,0}
\definecolor{purple}{rgb}{0.5,0,0.5}
\definecolor{darkred}{rgb}{0.5,0,0}
\definecolor{orange}{rgb}{0.5,0.5,0}

\newcommand{\insertpython}[2]%
{%
    \lstset{
        language=python,
        basicstyle=\ttfamily\small\setstretch{1},
        stringstyle=\color{green},
        showstringspaces=false,
        alsoletter={1234567890},
        otherkeywords={\ , \}, \{},
        keywordstyle=\color{blue},
        emph={access,and,as,break,class,continue,def,del,elif,else,%
        except,exec,finally,for,from,global,if,import,in,is,%
        lambda,not,or,pass,print,raise,return,try,while,assert},
        emphstyle=\color{orange}\bfseries,
        emph={[2]self},
        emphstyle=[2]\color{gray},
        emph={[4]ArithmeticError,AssertionError,AttributeError,BaseException,%
        DeprecationWarning,EOFError,Ellipsis,EnvironmentError,Exception,%
        False,FloatingPointError,FutureWarning,GeneratorExit,IOError,%
        ImportError,ImportWarning,IndentationError,IndexError,KeyError,%
        KeyboardInterrupt,LookupError,MemoryError,NameError,None,%
        NotImplemented,NotImplementedError,OSError,OverflowError,%
        PendingDeprecationWarning,ReferenceError,RuntimeError,RuntimeWarning,%
        StandardError,StopIteration,SyntaxError,SyntaxWarning,SystemError,%
        SystemExit,TabError,True,TypeError,UnboundLocalError,UnicodeDecodeError,%
        UnicodeEncodeError,UnicodeError,UnicodeTranslateError,UnicodeWarning,%
        UserWarning,ValueError,Warning,ZeroDivisionError,abs,all,any,apply,%
        basestring,bool,buffer,callable,chr,classmethod,cmp,coerce,compile,%
        complex,copyright,credits,delattr,dict,dir,divmod,enumerate,eval,%
        execfile,exit,file,filter,float,frozenset,getattr,globals,hasattr,%
        hash,help,hex,id,input,int,intern,isinstance,issubclass,iter,len,%
        license,list,locals,long,map,max,min,object,oct,open,ord,pow,property,%
        quit,range,raw_input,reduce,reload,repr,reversed,round,set,setattr,%
        slice,sorted,staticmethod,str,sum,super,tuple,type,unichr,unicode,%
        vars,xrange,zip},
        emphstyle=[4]\color{purple}\bfseries,
        upquote=true,
        morecomment=[s][\color{lightgreen}]{"""}{"""},
        commentstyle=\color{red}\slshape,
        literate={>>>}{\textbf{\textcolor{darkred}{>{>}>}}}3%
                 {...}{{\textcolor{gray}{...}}}3,
        procnamekeys={def,class},
        procnamestyle=\color{blue}\textbf,
        framexleftmargin=1mm, framextopmargin=1mm, frame=single,
        rulesepcolor=\color{blue},
        breaklines=true,
    }
    \lstset{caption=#2}
    \lstinputlisting[language=Python]{#1}
}


\pagestyle{headings}


\title{Le package \texttt{bnf}}
\author{Londeix Raphaël}

\renewcommand\rmdefault{pbk}  % famille à utiliser pour du Roman
%\renewcommand\sfdefault{cmss} % famille à utiliser pour du Sans Serif
%\renewcommand\ttdefault{bmtt} % famille à utiliser pour du «machine à écrire»
%\renewcommand\bfdefault{bx}   % collection à utiliser pour du gras
%\renewcommand\mddefault{m}    % collection à utiliser pour du moyen
%\renewcommand\itdefault{it}   % forme à utiliser pour de l'italique
%\renewcommand\sldefault{sl}   % forme à utiliser pour du penché
%\renewcommand\scdefault{sc}   % forme à utiliser pour de petites majuscules
%\renewcommand\updefault{n}    % forme à utiliser pour du droit
%\renewcommand\encodingdefault{OT1}      % codage normal
\renewcommand\familydefault{\rmdefault} % famille normale: Roman
%\renewcommand\seriesdefault{\mddefault} % collection normale: moyen
%\renewcommand\shapedefault{\updefault}  % forme normale: droit

\newcommand{\fixed}[1]{\texttt{#1}}
\newcommand{\bnf}{B.N.F.}

\begin{document}

    \maketitle

    \begin{abstract}
        Le package \fixed{bnf} fournit plusieurs classes permettant
        l'écriture de \textit{parser} la plus proche possible de la
        BNF sous-jacente.
    \end{abstract}

    \newpage
    \section{Introduction}
        Un besoin récurrent en informatique est l'extraction de contenu,
        qui n'est qu'en fait de la traduction. la \bnf (Backus Naur Form)
        permet de décrire la grammaire d'un language avec une syntaxe simple.
        Celle-ci permet de relier entre eux des \textit{tokens},
        parties du language donné, afin de former une définition globale
        de celui-ci. La syntaxe et le caractère dynamique de Python permettent
        une grande flexibilité lors de l'implémentation. Cette librairie se
        base sur cette flexibilité pour décrire la grammaire d'un language.
        Nous verrons en détails les classes de bases fournies et leurs
        utilisations. J'utiliserai
        abusivement certains anglissisme comme \emph{parser} pour désigner
        la hiérarchie des classes qui compose le parser, ou bien \emph{matching}
        lorsqu'il s'agira du processus d'identification des tokens.

    \newpage
    \section{Fonctionnement général}
        \subsection{La source de donnée}
            Pour pouvoir définir et utiliser les tokens, nous avons besoin d'une
            source de données. J'ai appelé cet objet le \emph{contexte} car
            il aggrège un certains nombre d'objet liés à l'état courant. le
            package \fixed{bnf} fournit une classe simple et prête à l'emploi.
            Celle-ci est utilisée par tous les tokens comme une source lors
            du \textit{matching}. Une instance de la classe \fixed{bnf.Context} est
            construites avec un nom de fichier en argument, fichier qui sera
            lu par les tokens.

            Initialiser un contexte est aussi simple que ça~:
            \insertpython{listings/bnf/ex00.py}{pif}

            Un exemple fonctionnel pourrait donner ceci~:
            \insertpython{listings/bnf/ex01.py}{pif}

            Vous noterez que je me force à mettre la notation \bnf avant chaque
            définition, cela se révèle très utile lors de la relecture. La ligne 10
            déclare \fixed{language} comme étant un \emph{littéral}, que la
            prochaine section décrit. Le seul fichier pouvant être parsé doit
            contenir exactement la chaîne "Hello, World!". En fait, il va
            accepter des caractères blancs avant et/ou après. Les tokens sont
            configuré pour extraires les caractères blanc usuels par défaut.
            Nous verrons comment modifier ce comportement, notemment lorsqu'il
            s'agira d'ignorer autre chose que des caractères (des commentaires
            par exemple).

        \subsection{Le concept de token}
            La classe \fixed{bnf.Token} défini les attributs et méthodes que
            doivent impérativement posséder ou redéfinir tous les tokens.
            La lecture du code source peut-être utile puisqu'il ne contient
            que très peu de code, et beaucoup de commentaire.
            Comme dit précédemment, la définition
            d'une grammaire résulte en une structure arborescente, la racine
            étant le language en lui-même, les feuilles étant les règles
            terminales de la \bnf.

            Chaque token possède une méthode \fixed{match}, prenant
            le contexte en argument et renvoyant un booléen. La valeur de ce booléen
            informe l'appelant que le token à \emph{matché} ou pas. L'écriture de
            cette méthode impose plusieurs contraintes afin que l'ensemble puisse
            fonctionner harmonieusement.
            Voici le comportement attendu par la méthode match~:
            \begin{itemize}
                \item Ne retourne rien d'autre que \fixed{True} ou \fixed{False}
                \item Laisse le contexte inchangé lorsque le retour est \fixed{False}
                \item Consomme effectivement les caractères lorsque le retour est \fixed{True}
                \item Appelle la méthode \fixed{onBeginMatch} avant toute autre chose
                \item Appelle la méthode \fixed{onMatch} lorsque le retour est \fixed{True}
                \item Appelle la méthode \fixed{onEndMatch} juste avant de retourner.
            \end{itemize}

            \noindent
            Les méthodes \fixed{onBeginMatch}, \fixed{onMatch} et
            \fixed{onEndMatch} ne font rien par défaut, elles doivent pouvoir être
            surdéfinies sans modifier le comportement général du token. Chacune
            de ces méthodes prend le contexte en argument, et \fixed{onEndMatch}
            prend en plus le résultat du matching en plus.

            Pour des raisons de commodité, tous les tokens possèdes une méthode
            \fixed{\_\_str\_\_} qui permet d'afficher les tokens dans leur
            représentation la plus fidèle possible à la \bnf.

    \newpage
    \section{Les classes de base}
        Nous allons ici décrire les classes les plus utiles pour créer un parseur.
        Tout d'abord, nous allons décrire les classes permettant de créer les
        règles terminales, puis les classes représentant les groupes (au sens de
        la \bnf) et enfin les classes qui nous simplifierons la vie.

        \subsection{Les chaînes de caractères}

            \subsubsection{La classe \fixed{bnf.Literal}}
                Nous avons abordé le concept de \emph{littéral} dans la première
                section. La classe utilisée ici est \fixed{bnf.Literal}, elle permet
                de \emph{matcher} une chaîne de caractère fixe. Reprennons le précédent
                exemple en utilisant une syntaxe un tout petit peu différente~:
                \insertpython{listings/bnf/ex02.py}{pif}

                Les lignes de 1 à 4 et de 10 à 14 seront souvent omisent dans les exemples suivants,
                nous nous concentrerons sur la \bnf. Ici, nous surclassons la
                classe \fixed{bnf.Literal}, ce qui ici n'apporte rien, mais sera
                la voie privilégiée par la suite. En effet, c'est elle
                qui permet de redéfinir les \emph{hooks} de matching
                (\fixed{onBeginMatch}, \fixed{onMatch} et \fixed{onEndMatch}).
                Les classes possèdent certains attributs spéciaux à connaître.
                Heureusement, ceux-ci sont en nombre très réduit et permettent
                d'accélérer grandement la surdéfinition des classes. Pour la classe
                \fixed{bnf.Literal}, c'est l'attribut \fixed{\_\_token\_\_} qui
                contient la chaîne à matcher.

            \subsubsection{La classe \fixed{bnf.Identifier}}
                Une autre classe permet de matcher des chaînes de caractères,
                c'est la classe \fixed{bnf.Identifier}. Celle-ci permet de
                matcher un \emph{pattern}, qui sera représenté par une expression
                régulière. Lorsqu'une chaîne est effectivement identifiée comme
                validant l'expression régulière, le membre \fixed{id} est
                initialisé avec la chaîne en question.
                Voici un exemple simple utilisant cette classe~:
                \insertpython{listings/bnf/ex03.py}{pif}

                Vous noterez cette fois que l'attribut particulier est
                \fixed{\_\_default\_regex\_\_}.

        \subsection{Les groupes et les alternatives}

            \subsubsection{La classe \fixed{bnf.Group}}
                Jusqu'a présent, les exemples ne montrent pas l'utilisation de
                plusieurs tokens. Nous allons maintenant pouvoir chaîner les
                tokens avec l'utilisation des groupes. Ceux-ci correspondent
                naturellement aux groupes tels que défini dans la \bnf. Il
                permettent de définir un token comme étant une séquence d'autres
                tokens, et de spécifier un nombre minimal et maximal de répétition.
                Voyons maintenant une série d'exemple de la classe \fixed{bnf.Group}~:
                \insertpython{listings/bnf/ex04.py}{pif}

                Les exemples parlent d'eux-même. Le constructeur d'un groupe prend
                en paramètre une liste. Il fait des traitements sur cette liste,
                afin de laisser au développeur une syntaxe la plus simple possible.
                Les éléments de type \fixed{str} sont automatiquement convertis en
                \fixed{bnf.Literal}. Vous pouvez aussi mettre dans la liste une
                \emph{classe}, et non pas une \emph{instance} de cette classe.
                Il est aussi possible de spécifier dans le constructeur le
                nombre minimal et maximal de répétition avec les paramètres \fixed{min}
                et \fixed{max}. Lorsque \fixed{max}
                vaut $-1$, cela signifie que le nombre de répétition est infini.
                Il est aussi possible de fournir des méthodes à la place d'entiers,
                qui seront alors appelées pour déterminer dynamiquement le nombre
                d'occurence minimum et maximum.


                Nous aurions aussi pu dériver la classe \fixed{bnf.Group}, par
                exemple~:
                \insertpython{listings/bnf/ex06.py}{pif}
                Cette fois, l'attribut est \fixed{\_\_group\_\_}. Il est initialisé
                avec une liste de token.

                Cette classe est en fait la classe la plus utilisée lors du développement
                d'un parser. Elle possède un certains nombre de possibilités, notament
                celle d'être récursive. Nous aborderons ces fonctionnalités dans le
                prochain chapitre.

            \subsubsection{La classe \fixed{bnf.Alternative}}
                J'appelle \emph{alternative} le «ou» de la \bnf. Encore une fois,
                la flexibilité de Python va nous faciliter la tâche. En effet, l'operateur
                binaire \fixed{|} est redéfini, et implique que nous n'ayons jamais
                besoin d'utiliser la classe explicitement. L'opérateur peut donc agir
                entre des instances ou des classes de façon totalement transparente.
                Mais voyons tout de suite quelques exemples~:
                \insertpython{listings/bnf/ex05.py}{pif}

                Au besoin, comme la classe \fixed{bnf.Alternative} dérive de \fixed{bnf.Group},
                elle est utilisable de la même façon.

        \subsection{Les classes utiles}
            Dans la vie réelle (prochaine section), nous aurons besoin de classes qui
            nous simplifierons la vie.

            \subsubsection{La classe \fixed{bnf.NamedToken}}
                Lorsque vous créez un groupe, chacun des tokens est contenu dans une liste
                qui ne vous est normalement pas accessible. Une manière plus claire pour
                récuperer un token reste de lui donner un nom. En effet, il n'est pas conseillé
                d'acceder au membre privé \fixed{\_group} ou encore de garder un pointeur vers
                le ou les tokens intéressant(s). Le but est d'avoir la syntaxe la plus
                claire possible. Pour récuperer le token, il suffit d'utiliser la méthode
                \fixed{getByName} de la classe \fixed{bnf.Group}.
                Voici un exemple permettant de parser un fichier contenant des paires clef, valeur~:
                \insertpython{listings/bnf/ex07.py}{pif}

                Vous remarquerez cepedant une petite lourdeur, la méthode \fixed{getByName}
                retourne en fait l'instance du \fixed{bnf.NamedToken}, et il faut ensuite
                utiliser \fixed{getToken} pour finalement récupérer les instances (ici, des
                classes \fixed{Key} et \fixed{Value}). la classe \fixed{bnf.NamedToken}
                possède aussi la méthode \fixed{getName}, si vous avez besoin de retrouver
                le nom.

            \subsubsection{La classe \fixed{bnf.TokenFunctor}}
                Nous avions évoqué la présence de différent \emph{hooks} (les méthodes \fixed{on[Begin|End]?Match}),
                mais il arrive assez souvent que ces hooks ne suffisent pas, ou rendent la tâche plus
                compliquées que ce qu'elle devrait être. La classe \fixed{bnf.TokenFunctor} est
                là pour pallier à ce besoin. Cette classe est un token comme les autres, à la différence qu'il
                renvoit le résultat d'une méthode à vous (toujours un booléen). Il est utile
                par exemple lors d'une alternative, ou bien lorsqu'un matching doit être plus intelligent.
                Par exemple, si on reprend l'exemple précédent et que l'on veut interdire la
                redéfinition de clef~:
                \insertpython{listings/bnf/ex08.py}{pif}

                Nous avons besoin de connaitre \fixed{self} pour pouvoir donner au \emph{functor} la
                méthode liée à l'instance courante, ce qui nous oblige à initialiser notre groupe dans
                le constructeur. On appelle juste le contructeur de la classe parente (\fixed{bnf.Group}),
                avec le la liste de token en paramètre. J'insiste sur le fait que la méthode passée
                au \emph{functor} doit retourner un booléen, même si elle ne valide rien (par
                exemple lors d'une initialisation lourde, placée le plus tard possible).


    \newpage
    \section{Le parser en condition réelle}
        Nous avons survolé presque toutes les classes du package \fixed{bnf}.
        Il est temps maintenant d'aborder des problèmes plus complexes, des
        problèmes inévitable dès lors que vous traduisez autre chose qu'un
        simple fichier CSV. Nous allons voir au travers d'exemples plus
        conséquents certaines de ces difficultées.

        \subsection{Les groupes récursifs}
            Les groupes récursifs sont monnaie courante, ils sont présent
            partout. Nous allons ici prendre l'exemple du format XML. Il est
            évident que quelle que soit la méthode choisie, nous n'allons pas
            définir différents types de balise en fonction du niveau d'arborescence
            à laquelle elles se trouvent. Nous allons suivre notre intuition,
            c'est à dire, une balise XML peut contenir soit du texte, soit
            une autre balise. C'est donc un groupe récursif.
            Voyons d'abord le code source~:
            \insertpython{listings/bnf/ex09.py}{pif}

            Il est nécessaire ici aussi de mettre la définition du groupe dans
            son constructeur car la classe elle même n'est pas encore définie.
        \subsection{Surclasser le contexte}
        \subsection{Vos propres tokens}

\end{document}
